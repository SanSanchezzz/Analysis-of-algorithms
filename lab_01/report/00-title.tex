
%\NirEkz{Экз. 3}                                  % Раскоментировать если не требуется
%\NirGrif{Секретно}                % Наименование грифа

%\gosttitle{Gost7-32}       % Шаблон титульной страницы, по умолчанию будет ГОСТ 7.32-2001,
% Варианты GostRV15-110 или Gost7-32

\NirOrgLongName{
МОСКОВСКИЙ ГОСУДАРСТВЕННЫЙ ТЕХНИЧЕСКИЙ УНИВЕРСИТЕТ ИМ. Н. Э. БАУМАНА
}                                           %% Полное название организации

\NirUdk{УДК № 004.822}
%\NirGosNo{№ госрегистрации }
%\NirInventarNo{Инв. № ??????}

%\NirConfirm{Согласовано}                  % Смена УТВЕРЖДАЮ
\NirBoss[.49]{Преподаватель}{}            %% Заказчик, утверждающий НИР


%\NirReportName{Научно-технический отчет}   % Можно поменять тип отчета
\NirAbout{По лабораторной работе №1} %Можно изменить о чем отчет

%\NirPartNum{Часть}{1}                      % Часть номер

\NirBareSubject{}                  % Убирает по теме если раскоментить

% \NirIsAnnotacion{АННОТАЦИОННЫЙ }         %% Раскомментируйте, если это аннотационный отчёт
%\NirStage{промежуточный}{Этап \No 1}{} %%% Этап НИР: {номер этапа}{вид отчёта - промежуточный или заключительный}{название этапа}
%\NirStage{}{}{} %%% Этап НИР: {номер этапа}{вид отчёта - промежуточный или

\Nir{}

\NirSubject{Редакционное расстояние}                                   % Наименование темы
%\NirFinal{}                        % Заключительный, если закоментировать то промежуточный
%\finalname{итоговый}               % Название финального отчета (Заключительный)
%\NirCode{Шифр\,---\,САПР-РЛС-ФИЗТЕХ-1} % Можно задать шифр как в ГОСТ 15.110
\NirCode{}

\NirManager{Студент}{А. А. Куприй} %% Название руководителя
\NirIsp{Преподаватели}{Л.Л. Волкова, Ю.В. Строганов} %% Название руководителя

% \NirYear{1999}%% если нужно поменять год отчёта; если закомментировано, ставится текущий год
\NirTown{Москва}                           %% город, в котором написан отчёт
