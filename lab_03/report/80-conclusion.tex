\Conclusion % заключение к отчёту
В ходе даннои работы были изучены алгоритмы поразрядной и быстрой сортировок, сортировки
вставками; были получены практические навыки реализации указанных алгоритмов. Была проведена
оценка сложности теоретически с указанием лучшего и худшего случаев (если есть) и условий их наступления. Был проведен сравнительный анализ перечисленных алгоритмов сортировки по затрачиваемым
ресурсам времени и получено экспериментальное подтверждение различий во временн´oй эффективности
выбранных алгоритмов сортировки при помощи разработанного программного обеспечения на материале
замеров процессорного времени выполнения реализации на варьирующихся длинах массивов. В результате были получены следующие выводы:
\begin{enumerate}

	\item Сортировка вставками работает медленнее остальных исследуемых алгоритмов при во всех рассмотренных случаях на длинных массивах (≈ от 100 элементов). Но этот алгоритм эффективен на
небольших наборах данных, на наборах данных до десятков элементов может оказаться лучшим;
он также эффективен на наборах данных, которые уже частично отсортированы; это устойчивый
алгоритм сортировки (не меняет порядок элементов, которые уже отсортированы);
	\item Основным достоинством поразрядной сортировки является скорость, однако она требует использования дополнительной памяти и имеет узкую специализацию;
	\item	Алгоритм быстрой сортировки является одним из самых быстрых универсальных алгоритмов сортировки массивов. Однако в худшем случае глубина рекурсии при выполнении алгоритма достигнет n,
что будет означать n-кратное сохранение адреса возврата и локальных переменных процедуры разделения массивов. Для больших значений n худший случай может привести к исчерпанию памяти
(переполнению стека) во время работы программы.
\end{enumerate}
